The number of division events in a growing colony up to time \( t \) is approximately:
\begin{equation}
    D(t) = \int_{0}^{t} N(\tau) d\tau \approx \frac{N_{0}}{\lambda} e^{\lambda t}
    \label{eq:num-divisions}
\end{equation}
The number of possible base-pair mutations in these events is:
\begin{equation}
    n(t) = D(t) L
    \label{eq:n-binomial}
\end{equation}
Thus, the number of mutants in the colony at time \( t \) is a binomial RV:
\begin{equation}
    M(t) \sim \text{Binomial}(n(t), \mu)
    \label{eq:naive-binomial}
\end{equation}
We know that the mean and variance of a binomial RV are:
\begin{equation}
    \text{E}(M(t)) = n(t) \mu
    \label{eq:mean-naive-binomial}
\end{equation}
\begin{equation}
    \text{Var}(M(t)) = n(t) \mu (1 - \mu)
    \label{eq:var-naive-binomial}
\end{equation}
Thus the Fano factor of the number of mutants in the colony at time \( t \) is:
\begin{equation}
    F(M(t)) = \frac{\text{Var}(M(t))}{\text{E}(M(t))} = 1 - \mu
    \label{eq:naive-fano}
\end{equation}
For any possible static distribution, it is not feasible that, to set some arbitrary lower limit, \(\mu > 10 ^ {-2}\), for biological reasons.
This would mean that we would have mutations in every gene, including essential genes, and these mutators would die out almost instantly.
Thus, we can safely say that, in this case, \(1-\mu \ll 1\) and:
\begin{equation}
    F(M(t)) \approx 1
    \label{eq:naive-fano-approx}
\end{equation}