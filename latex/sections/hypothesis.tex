In all the following system we assume a growth rate $\lambda$ that does not change between mutants,
such that population size at time \(t \) is always:
\begin{equation}
    N(t) = N_{0} e^{\lambda t}
    \label{eq:growth}
\end{equation}
We assume a per-base pair mutation rate, with a genome of length \(L\).

\subsection{Static Distribution}\label{subsec:static}
    The most simple and yet still plausible control system, is one in which each individual in the population draws it's mutation rate
    from a static distribution with mean mutation rate $\mu$.
    This distribution is the same for all individuals and does not change over time.

\subsection{Hereditary Noise}\label{subsec:hereditary}
    In this model, we assume that the mutation rate of an individual is determined by the mutation rate of it's parent, with some additive noise.
    In effect, this means that the distribution of mutation rates in the population is a diffusion process.
    Biologically, one can assume there exists some \( \mu_c \) such that for any individual with mutation rate \( \mu > \mu_c \), it is likely not to survive.
    Of course, one must have that all \( \mu \geq 0 \) just from the definition of mutation rate.
    These interesting BC will be discussed later.

\subsection{Stochastic State Switching}\label{subsec:sss}
    \begin{enumerate}
        \item The system has 2 different states:
        \begin{itemize}
            \item State 1: Low mutator cells with mutation rate $\mu_{L}$
            \item State 2: High mutator cells with mutation rate $\mu_{H}$, where $\mu_{H} \gg \mu_{L}$.
        \end{itemize}
        \item There exist switching rates between these states:
            \begin{itemize}
                \item $r_{LH}$ : Rate of switching from State 1 to State 2
                \item $r_{HL}$ : Rate of switching from State 2 to State 1
            \end{itemize}
        \item These rates result in equilibrium fractions of high and low mutator cells:
            \begin{itemize}
                \item Equilibrium fraction of high mutator cells: $\hat{f}_{H} = \frac{r_{LH}}{r_{LH} + r_{HL}}$
                \item Equilibrium fraction of low mutator cells: $\hat{f}_{L} = 1 - \hat{f}_{H} = \frac{r_{HL}}{r_{LH} + r_{HL}}$
            \end{itemize}
        \item $\mu = \hat{f}_{H} \mu_{H} + \hat{f}_{L} \mu_{L}$ is the average mutation rate for a stochastic switching colony.
    \end{enumerate}
    So here we have a phenotypic state switching system, where the state is hereditary.
    The switching rates create an average correlation time per lineage.
    This kind of system has been observed in the different context of persistent cells, perhaps because of its stability -
    even if any section of the population perishes, the surviving population equilibrates to the same(perhaps ideal) equilibrium fractions.
    For the time dependent solution of the fraction of high mutator cells in the colony, see~\ref{sec:sss}.


